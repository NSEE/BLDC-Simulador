% ------------------------------------------------------------------------
%  Instituto Mauá de Tecnologia
%  Núcleo de Sistemas Eletrônicos Embarcados - NSEE
%   Juliano Tusi Amaral Laganá Pinto
% 					 julianotusi@gmail.com
%  		
%  Projeto CITAR
%	Meta 4
% 	Especificação do produto
% 	Acionamento motor BLDC para rodas de reação
%
%	Outubro de 2014
% ------------------------------------------------------------------------
\input{template_nota}
\usepackage{natbib}
%------------------------------------------------------------------------	
\usepackage{tikz}
\graphicspath{ {./figs/} }
% ------------------------------------------------------------------------
\newtcolorbox[auto counter,number within=section]{TBD-BOX}[2][]{%
colback=red!5!white,colframe=darkgray,fonttitle=\bfseries,
title=TBD.~\thetcbcounter: #2,#1}
% ------------------------------------------------------------------------
% http://latexcolor.com/
% cores dos todos
\definecolor{TBC}{rgb}{0.98, 0.91, 0.71}  	%bananamania
\definecolor{Corsi}{rgb}{0.53, 0.66, 0.42}  %
%\definecolor{TBC2}{rgb}{1.0, 0.71, 0.76} 		%lightpink
\definecolor{TBC2}{rgb}{0.86, 0.82, 1.0} %lightmauve
\definecolor{darkgray}{rgb}{0.66, 0.66, 0.66}
\definecolor{aliceblue}{rgb}{0.94, 0.97, 1.0}
% ------------------------------------------------------------------------

\title		{   \LARGE
				Instituto Mauá de Tecnologia \\
				Núcleo de Sistemas Eletrônicos Embarcados - NSEE \\
				\vspace{10px}
				\huge
				Simulador motor BLDC \\
			}
\author		{
				 Juliano Tusi Amaral Laganá Pinto \\ \small 	
				\href{mailto:julianotusi@gmail.com}{julianotusi@gmail.com}
			}
%\subject	{Núcleo de Sistemas Eletrônicos Embarcados - NSEE}			
\titlepic	{\includegraphics[width=0.4\textwidth]{maua_logo}}
%\lowertitleback{}

% ------------------------------------------------------------------------
\begin{document}
% ------------------------------------------------------------------------
\maketitle
\tableofcontents
% ------------------------------------------------------------------------
%%---------- Revisão
\begin{versionhistory}
  \vhEntry{1.0.0}{16.10.14}{Juliano Laganá}{Criação do documento.}
  \vhEntry{1.0.1}{21.10.14}{Juliano Laganá}{Adição do encoder e do método M/T aos blocos especificados.}
  \vhEntry{1.0.2}{24.06.15}{Juliano Laganá}{Substituição do modelo elétrico de fases individuais pelo modelo elétrico com as fases acopladas.}
  \vhEntry{1.0.3}{24.06.15}{Juliano Laganá}{Adição do bloco de conversão de tensões.}
  \vhEntry{1.0.4}{24.06.15}{Juliano Laganá}{Correção da definição do parâmetro L. Na verdade é a indutância por fase descontada da indutância mútua.}
  \vhEntry{1.0.5}{25.06.15}{Juliano Laganá}{Inclusão do modelo de atrito de Coulomb}
  \vhEntry{1.0.6}{29.06.15}{Juliano Laganá}{Atualização: agora o usuário pode definir o formato de back-emf que quiser}
  \vhEntry{1.0.7}{30.06.15}{Juliano Laganá}{Atualização do bloco de comutação six step: agora ele está desacoplado do BLDC}
\end{versionhistory}



\chapter{Introdução}

    \section{Objetivo}
        O objetivo desse documento é disponibilizar um manual de uso da biblioteca ``BLDC.slx'' para simulação do motor DC sem escovas (BLDC).

    \section{Sistema de interesse}
        O sistema de interesse é um motor DC sem escovas acoplado à uma carga constante. A figura \ref{fig_modelo_simplificado} ilustra um modelo simplificado do sistema. Os torques mostrados no desenho são:
        \begin{itemize}
            \item $T_e$ : Torque elétrico gerado pelo motor.
            \item $T_d$ : Torque de atrito no mancal (atrito viscoso + atrito de Coulomb).
            \item $T_l$ : Torque da carga.
        \end{itemize}

        \begin{figure}[ht]
            \centering
            \includegraphics[width=0.7\textwidth]{sketch_bldc}
            \caption{Modelo simplificado do sistema}
            \label{fig_modelo_simplificado}
        \end{figure}

    \section{Hipóteses}
        \begin{itemize}
            \item Todas as fases de alimentação do BLDC tem a mesma resistência.
            \item Todas as fases de alimentação do BLDC tem a mesma indutância.
            \item Todos as partes do sistema são consideradas corpos rígidos.
            \item O atrito no mancal pode ser modelado como atrito viscoso (diretamente proporcional à velocidade angular do rotor em relação ao estator) + atrito de Coulomb.
        \end{itemize}

    \section{Modelo matemático}
    \label{sec:modelo_matematico}
        \begin{itemize}
            \item Dinâmica mecânica, com atrito viscoso e atrito de Coulomb:
                $$J.\frac{d^2\theta_m}{dt^2}=T_e-T_l-T_{at1}-T_{at2}$$
                $$T_{at1} = K_d.\frac{d\theta_m}{dt}$$
                 \begin{displaymath}
                   T_{at2} = \left\{
                     \begin{array}{lr}
                       -T_k.sign(\frac{d\theta}{dt}) & : \frac{d\theta_m}{dt} \neq 0\\
                       -min(T_s,T_e-T_l).sign(T_e-T_l) & : \frac{d\theta_m}{dt} = 0
                     \end{array}
                   \right.
                \end{displaymath}
            \item Dinâmica elétrica, como proposto por \cite{baldursson}:
                $$v_{ab} = R(i_a-i_b)+L\frac{d}{dt}(i_a-i_b)+e_a-e_b$$
                $$v_{bc} = R(i_a+2i_b)+L\frac{d}{dt}(i_a+2i_b)+e_b-e_c$$
                $$i_a+i_b+i_c=0$$
            \item Forças contraeletromotrizes geradas em cada fase \cite{baldursson}:
                $$e_a = \frac{f_a(\theta_e).K_e}{2}.\frac{d\theta_m}{dt} $$
                $$e_b = \frac{f_b(\theta_e).K_e}{2}.\frac{d\theta_m}{dt} $$
                $$e_c = \frac{f_c(\theta_e).K_e}{2}.\frac{d\theta_m}{dt} $$
            \item Torque gerado por cada fase e torque elétrico total \cite{baldursson}:
                $$T_a=\frac{f_a(\theta_e).K_t.i_a}{2}$$
                $$T_b=\frac{f_b(\theta_e).K_t.i_b}{2}$$
                $$T_c=\frac{f_c(\theta_e).K_t.i_c}{2}$$
                $$T_e=T_a+T_b+T_c$$
            \item Relação entre ângulo elétrico e ângulo mecânico:
                $$\theta_e=\theta_m.\frac{P}{2}$$
        \end{itemize}

        \begin{description}
            \item[$J$] = Inércia do sistema motor+carga
            \item[$K_d$] = Coeficiente de atrito viscoso do mancal
            \item[$T_{at1}$] = Torque devido ao atrito viscoso
            \item[$T_{at2}$] = Torque devido ao atrito de Coulomb
            \item[$T_k$] = Magnitude do torque devido ao atrito cinético
            \item[$T_s$] = Magnitude máxima do torque devido ao atrito estático

            \item[$K_e$] = Constante de força contraeletromotriz
            \item[$K_t$] = Constante de torque
            \item[$P$] = Número de pólos

            \item[$\theta_m$] = Ângulo do rotor em relação ao estator
            \item[$\theta_e$] = Ângulo elétrico

            \item[$L$] = Indutância de cada fase (descontada do valor da indutância mútua, caso seja significativo)
            \item[$R$] = Resistência de cada fase
            \item[$v_{ab}$] = Tensão entre as fase $a$ e $b$
            \item[$v_{bc}$] = Tensão entre as fases $b$ e $c$
            \item[$i_a$] = Corrente na fase $a$
            \item[$i_b$] = Corrente na fase $b$
            \item[$i_c$] = Corrente na fase $c$
            \item[$e_a$] = Força contraeletromotriz gerada na fase $a$
            \item[$e_b$] = Força contraeletromotriz gerada na fase $b$
            \item[$e_c$] = Força contraeletromotriz gerada na fase $c$
            \item[$T_a$] = Torque elétrico gerado pela fase $a$ no rotor
            \item[$T_b$] = Torque elétrico gerado pela fase $b$ no rotor
            \item[$T_c$] = Torque elétrico gerado pela fase $c$ no rotor
            \item[$T_e$] = Torque elétrico total aplicado no rotor

            \item[$f_a(\theta_e)$] = Função que reproduz o comportamento da força contraeletromotriz na fase $a$ (padrão: comportamento trapezoidal)
            \item[$f_b(\theta_e)$] = Função que reproduz o comportamento da força contraeletromotriz na fase $b$ (padrão: comportamento trapezoidal)
            \item[$f_c(\theta_e)$] = Função que reproduz o comportamento da força contraeletromotriz na fase $c$ (padrão: comportamento trapezoidal)

            \item[$sign(x)$] = Retorna 1 se $x$ é não-negativo, $-1$ caso contrário.
            \item[$min(a,b)$] = Retorna $a$ se $a>b$, $b$ caso contrário.


        \end{description}

\chapter{Implementação}

    \section{Bloco BLDC}
        O modelo matemático apresentado na seção \ref{sec:modelo_matematico} foi implementado no bloco ``BLDC'' ilustrado na figura \ref{fig:bloco_BLDC}. Todos os parâmetros do motor podem ser alterados clicando-se duas vezes em cima do bloco.

        Caso o usuário queira, ao invés de utilizar o formato padrão trapezoidal para o formato das forças contra-eletromotrizes, ele pode especificar um formato diferente nos parâmetros do bloco. Para tal, é necessário informar três matrizes $n\times2$ que servirão de lookup table. A primeira coluna dessas matrizes é o tempo (deve começar em zero, ser monotonicamente crescente e terminar em 360) e a segunda coluna é o valor de $f(\theta_e)$ para aquela fase.

        Exemplo: \texttt{backemfa = [linspace(0,360,200)',sin(linspace(0,360,200)')];}

        \textbf{Atenção:} desde a adição do modelo de atrito de Coulomb, esse bloco não suporta mais simulação com fixed-step solvers. Caso um fixed-step solver seja utilizado, uma mensagem de erro será apresentada ao usuário.
        \begin{figure}[ht]
            \centering
            \includegraphics[width=0.3\textwidth]{bloco_BLDC}
            \caption{Bloco simulink para simulação do BLDC}
            \label{fig:bloco_BLDC}
        \end{figure}

        \begin{itemize}
            \item Entradas
            \begin{enumerate}
                \item Vab - Tensão entre as fases $a$ e $b$ [V]
                \item Vbc - Tensão entre as fases $b$ e $c$ [V]
                \item Tl - Torque da carga [N.m]
            \end{enumerate}
            \item Saídas
            \begin{enumerate}
                \item theta\_m - Ângulo entre o rotor e o estator ($\theta_m$) [rad]
                \item w\_m - Velocidade angular entre o rotor e o estator ($\frac{d\theta_m}{dt}$) [rad/s]
                \item internal - Sinal multiplexado [4x1] composto pelos seguintes sinais:
                \begin{itemize}
                    \item Correntes [3x1] - $i_a$, $i_b$ e $i_c$ [A]
                    \item Torques [3x1] - $T_a$, $T_b$ e $T_c$ [N.m]
                    \item FCEMs [3x1] - $e_a$, $e_b$ e $e_c$ [V]
                    \item Torque total - $T_e$ [N.m]
                    \item Torque de atrito - $T_{at2}$
                \end{itemize}
            \end{enumerate}
        \end{itemize}

    \section{Bloco BLDC com lógica de comutação em blocos}
        Para o devido funcionamento do BLDC é necessário comutar a tensão entre cada fase periodicamente. O bloco apresentado na figura \ref{fig:bloco_BLDC_block_commutation} implementa internamente a estratégia de comutação em blocos através da utilização de três sensores de efeito hall separados por 120 graus. Os parâmetros da comutação (número de pólos do motor que será comutado) podem ser alterados clicando-se duas vezes em cima do bloco.
        \begin{figure}[ht]
            \centering
            \includegraphics[width=0.3\textwidth]{bloco_acionamento_six_step}
            \caption{Bloco simulink que implementa a comutação em blocos}
            \label{fig:bloco_BLDC_block_commutation}
        \end{figure}

        \begin{itemize}
            \item Entradas
            \begin{enumerate}
                \item V - Tensão que será aplicada nas fases quando forem acionadas [V]
                \item theta\_m - Essa porta deve ser conectada na saída theta\_m do simulador BLDC [rad/s]
            \end{enumerate}
            \item Saídas
            \begin{enumerate}
                \item Va - Tensão que deve ser aplicada na fase $a$ [V]
                \item Vb - Tensão que deve ser aplicada na fase $b$ [V]
                \item Vc - Tensão que deve ser aplicada na fase $c$ [V]
                \item Halls - Vetor [3x1] que contém os níveis lógicos de cada sensor Hall acoplado ao motor ($H_1$, $H_2$ e $H_3$)
            \end{enumerate}
        \end{itemize}

\chapter{Ferramentas adicionais}

    \section{Conversão de tensões}
        Foi desenvolvido um bloco para converter as tensões individuais de cada fase ($V_a$, $V_b$ e $V_c$) nas tensões entre as fases ($v_ab$ e $v_bc$, que são as entradas do bloco BLDC), ilustrado na figura \ref{fig:bloco_conversao_tensoes}.
        \begin{figure}[ht]
            \centering
            \includegraphics[width = 0.15\textwidth]{bloco_conversao_tensoes}
            \caption{Bloco para converter as tensões individuais em tensões entre fases}
            \label{fig:bloco_conversao_tensoes}
        \end{figure}

        \begin{itemize}
            \item Entradas
                \begin{enumerate}
                    \item $Va$ - Tensão na fase $a$
                    \item $Vb$ - Tensão na fase $b$
                    \item $Vc$ - Tensão na fase $c$
                \end{enumerate}
            \item Saídas
                \begin{enumerate}
                    \item $v_{ab}$ - Tensão entre as fases $a$ e $b$
                    \item $v_{bc}$ - Tensão entre as fases $b$ e $c$
                \end{enumerate}
        \end{itemize}

    \newpage
    \section{Medição de posição}
        Para medição de posição angular do rotor foi desenvolvido um bloco que simula o funcionamento de um encoder de quadratura, ilustrado na figura \ref{fig:bloco_encoder}. O número de pulsos por revolução pode ser configurado clicando-se duas vezes no bloco.
        \begin{figure}[ht]
            \centering
            \includegraphics[width=0.3\textwidth]{bloco_encoder}
            \caption{Bloco simulink para simulação de um encoder de quadratura}
            \label{fig:bloco_encoder}
        \end{figure}

        \begin{itemize}
            \item Entradas
                \begin{enumerate}
                    \item Theta - Ângulo mecânico entre o rotor e o estator. [graus]
                \end{enumerate}
            \item Saídas
                \begin{enumerate}
                    \item Theta\_amostrado - Ângulo mecânico entre o rotor e o estator, amostrado pelo encoder de quadratura simulado. [graus]
                \end{enumerate}
            \item Restrições
                \begin{enumerate}
                    \item Para o correto funcionamento desse bloco, o step size da simulação deve ser configurado para nunca exceder $\frac{360}{4.N_r.V_{máx}}$ segundos; onde $N_r$ é o número de pulsos por rotação do encoder (especificado clicando-se duas vezes em cima do bloco) e $V_{máx}$ é o valor máximo da derivada da entrada Theta em graus/s. Exemplo: Para a correta simulação de um encoder com $N_r=300$, amostrando um BLDC cuja velocidade angular atinge no máximo $1200$ graus/s (portanto o valor máximo da derivada da entrada Theta também é $1200$ graus/s), faz-se necessário configurar a simulação para que o step size nunca exceda $2,5 . 10^{-4}$ segundos.
                \end{enumerate}
        \end{itemize}


    \newpage
    \section{Medição de velocidade}
        O algoritmo M/T para estimação de velocidade \cite{algoritmoMT} foi implementado no simulink e está ilustrado na figura \ref{fig:bloco_MT}
        \begin{figure}[ht]
            \centering
            \includegraphics[width=0.3\textwidth]{bloco_MT}
            \caption{Bloco que implementa o algoritmo de estimação de velocidade M/T}
            \label{fig:bloco_MT}
        \end{figure}

        \begin{itemize}

            \item Entradas
                \begin{enumerate}
                    \item Posicao - Essa entrada deve ser conectada à saída ``Theta\_amostrado'' do bloco simulador de encoder de quadratura. [graus]
                \end{enumerate}
            \item Saídas
                \begin{enumerate}
                    \item Velocidade - Velocidade estimada pelo algorimo M/T. [graus/s]
                \end{enumerate}
        \end{itemize}




\bibliographystyle{plain}
\bibliography{bibfile}


\end{document}
