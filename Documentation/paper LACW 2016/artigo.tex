\documentclass{article}

\usepackage{graphicx}
\usepackage{multicol}
\usepackage{matlab-prettifier}
\usepackage{graphicx}
\usepackage{subcaption}
\usepackage{float}
\usepackage{verbatimbox}
\usepackage{amssymb}
\usepackage{amsmath}
\usepackage{epstopdf}

\usepackage{tikz}
\usepackage{pgfplots}

% For drawing circuits
\usepackage{circuitikz}
\ctikzset {bipoles/length=1.2cm}

\pgfplotsset{compat=newest}
\usepgfplotslibrary{units}
\usetikzlibrary{plotmarks}

% decrease all margins
\addtolength{\oddsidemargin}{-1cm}
\addtolength{\evensidemargin}{-1cm}
\addtolength{\textwidth}{2cm}

% definition to control the margins for a given part of text
\def\changemargin#1#2{\list{}{\rightmargin#2\leftmargin#1}\item[]}
\let\endchangemargin=\endlist

% all matlab code should appear numbered and with a left line separating line numbers from code
\lstset{numbers = left, frame=single}

% graphics are in 'images' folder
\graphicspath{{./images/}}

% useful for development
\usepackage{color}


\begin{document}

    \newpage

    The first step to obtain a model of the system's dynamic behavior was to separate it into two smaller coupled subsystems. Because of the electromechanical characteristic of BLDCs, it's natural to separate it into an electrical and a mechanical part.

    All of the electric and magnetic parts of the system (e.g., windings, coils) are lumped together in the electrical subsystem,
    to which the inputs are the voltages in each of the phases and the output is the electromagnetic torque that will be applied to the shaft.  On the other hand, the mechanical parts and variables --- such as angular displacement, inertia, friction --- are part of the mechanical subsystem, with the electromagnetic torque as input and the angular velocity of the motor's shaft as its output.

    The following sections analyze each of these subsystems in more detail, in order to obtain the mathematical relations between the relevant variables.


    \subsection{Electrical subsystem analysis}
    \label{sec:elec_subsystem}

    A diagram of the electrical subsystem is shown in figure \ref{fig:electrical_subsys}, assuming that all the phases have the same inductances and resistances. From it, it's possible to obtain expressions for $v_{ab}$, $v_{bc}$ and $v_{ca}$ (defined as $v_a-v_b$, $v_b-v_c$ and $v_c-v_a$, respectively) \cite[p. 457-459]{ref:electrical_modeling}. That, added to the fact that the currents in each phase must sum to zero (Kirchhoff's current law applied at the central node) and assuming that the mutual inductances are negligible, leads to the following equations:

    \begin{figure}[H]
        \center
        \newcommand{\myvar}{1.8}
        \begin{circuitikz}[scale=0.6, every node/.style={scale=1}]
            \ctikzset{label/align = straight}
            \draw

            % Phase a
            (0,3*\myvar) to [R=$R$] (0,2*\myvar) to [american voltage source, l_=$e_a$] (0,1*\myvar) to [inductor, l=$L$, -*] (0,0)
            (0,3*\myvar+1) to [short,i=$i_a$, o-] (0,3*\myvar)
            (0,3*\myvar+1) node[anchor=south]{$a$}

            % Phase b
            (-\myvar*3*0.866,-\myvar*3/2) to [R=$R$] (-\myvar*2*0.866,-\myvar*2/2) to [american voltage source, l=$e_b$] (-\myvar*0.866,-\myvar/2) to [inductor, l=$L$] (0,0)
            (-\myvar*3*0.866-1*0.866,-\myvar*3/2-1/2) to [short, i=$i_b$,o-] (-\myvar*3*0.866,-\myvar*3/2)
            (-\myvar*3*0.866-1*0.866,-\myvar*3/2-1/2) node[left]{$b$}

            % Phase c
            (\myvar*3*0.866,-\myvar*3/2) to [R,l_=$R$] (\myvar*2*0.866,-\myvar*2/2) to [american voltage source, l_=$e_c$] (\myvar*0.866,-\myvar/2) to [inductor, l,l_=$L$] (0,0)
            (\myvar*3*0.866+1*0.866,-\myvar*3/2-1/2) to [short,i=$i_c$,o-] (\myvar*3*0.866,-\myvar*3/2)
            (\myvar*3*0.866+1*0.866,-\myvar*3/2-1/2) node[right]{$c$}

            ;
        \end{circuitikz}
        \caption{Diagram of the electrical circuit of a three-phase BLDC motor}
        \label{fig:electrical_subsys}
    \end{figure}

    \begin{equation}
        v_{ab} = R(i_a-i_b)+L\frac{d}{dt}(i_a-i_b)+e_a-e_b
        \label{eq:v_ab}
    \end{equation}

    \begin{equation}
        v_{bc} = R(i_a+2i_b)+L\frac{d}{dt}(i_a+2i_b)+e_b-e_c
    \end{equation}

    \begin{equation}
        i_a+i_b+i_c=0.
    \end{equation}

    Additionally, since the back-emfs in each phase are proportional to the angular velocity, we can write them as

    \begin{equation}
        e_a = \frac{f_a(\theta_e).K_e}{2}.\frac{d\theta_m}{dt}
        \label{eq:e_a}
    \end{equation}

    \begin{equation}
        e_b = \frac{f_b(\theta_e).K_e}{2}.\frac{d\theta_m}{dt}
        \label{eq:e_b}
    \end{equation}

    \begin{equation}
        e_c = \frac{f_c(\theta_e).K_e}{2}.\frac{d\theta_m}{dt}.
        \label{eq:e_c}
    \end{equation}

    Here, $K_e$ is the motor's back-emf constant and $f_a(\theta_e)$, $f_b(\theta_e)$ and $f_c(\theta_e)$ are the functions that give the intended waveform for the back-emf; $\theta_m$ is the rotor's angular position in relation to the stator and $\theta_e$ is the electrical angular position, defined as

    \begin{equation}
        \theta_e = \theta_m.\frac{P}{2},
    \end{equation}

    where $P$ is the motor's number of poles.

    Assuming constant a constant air gap in the motor, an analysis of the consumed power of the machine yields the electromagnetic torque expression \cite[p. 459]{ref:electrical_modeling}. This can be further simplified by substituting equations \ref{eq:e_a} through \ref{eq:e_c} in the result. The final expression is

    \begin{equation}
        T_e = \frac{K_t}{2}.\Big(f_a(\theta_e).i_a + f_b(\theta_e).i_b + f_c(\theta_e).i_c\Big).
        \label{eq:total_torque}
    \end{equation}

    Equations \ref{eq:v_ab} through \ref{eq:total_torque} model the dynamic relationship between the electromagnetic torque and the currents, back-emfs and voltages of each phase of the BLDC.




    \subsection{Mechanical subsystem analysis}

    A diagram for the mechanical subsystem is illustrated in figure \ref{fig:mechanical_subsys} (to improve clarity, the load is not shown in this diagram).

    \begin{figure}[H]
        \center
        \begin{tikzpicture}
            \node[anchor=south west, inner sep = 0] (image) at (0,0) {\includegraphics[width = 0.4\linewidth]{mechanical_diagram}};
            \begin{scope}[x={(image.south east)},y={(image.north west)}]
            %    \draw[help lines,xstep=.1,ystep=.1] (0,0) grid (1,1);
            %    \foreach \x in {0,1,...,9} { \node [anchor=north] at (\x/10,0) {0.\x}; }
            %    \foreach \y in {0,1,...,9} { \node [anchor=east] at (0,\y/10) {0.\y}; }
                \draw (0.7,0.125) node {$T_e$};
                \draw (0.82,0.85) node {$T_f$};
                \draw (0.95,0.85) node {$T_l$};
            \end{scope}
        \end{tikzpicture}
        \caption{Mechanical subsystem}
        \label{fig:mechanical_subsys}
    \end{figure}

    Applying Newton's second law to the motor's rotor, we obtain the following equation

    \begin{equation}
        J.\frac{d^2\theta_m}{dt^2}=T_e-T_f-T_l,
        \label{eq:newtons_law_at_shaft}
    \end{equation}

    where $J$ is the motor's rotor inertia, $T_f$ is the resistance torque due to friction and $T_l$ is the torque applied by the load to the rotor.

    In order to provide an accurate motor model for a wide range of angular velocities, not only viscous friction must considered in the model. Therefore, $T_f$ is decomposed as viscous friction (referred to as $T_{f1}$) and Coulomb friction (referred to as $T_{f2}$)

    \begin{equation}
        T_f = T_{f1} + T_{f2}.
    \end{equation}

    Viscous frictions can be calculated with

    \begin{equation}
        T_{f1} = K_d.\frac{d\theta_m}{dt},
    \end{equation}

    where $K_d$ is the damping constant.

    And Coulomb's model of friction \cite[p.171]{ref:coulomb_friction} is given by

    \begin{equation}
       T_{f2} = \left\{
         \begin{array}{lr}
           -T_k.sign(\frac{d\theta}{dt}) & : \frac{d\theta_m}{dt} \neq 0\\
           -min(T_s,T_e-T_l).sign(T_e-T_l) & : \frac{d\theta_m}{dt} = 0
         \end{array}
       \right.
       \label{eq:coulomb_friction_model}
    \end{equation}

    where $T_k$ is the static friction constant and $T_s$ is the kinetic friction constant.

    Analogously to section \ref{sec:elec_subsystem}, equations \ref{eq:newtons_law_at_shaft} through \ref{eq:coulomb_friction_model} model the dynamic relationship between the electromagnetic torque and the angular position of the rotor. This equations, together with equations \ref{eq:v_ab} through \ref{eq:total_torque}, constitute the full mathematical model used to simulate the BLDC.


    \bibliographystyle{plain}
    \bibliography{bibliography}


\end{document}

